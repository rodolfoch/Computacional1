\documentclass[12pt]{article}
\usepackage[spanish]{babel}
%\usepackage{natbib}
\usepackage{url}
\usepackage[utf8x]{inputenc}
\usepackage{amsmath}
\usepackage{graphicx}
\graphicspath{{images/}}
\usepackage{parskip}
\usepackage{float}
\usepackage{fancyhdr}
\usepackage{vmargin}
\usepackage[usenames]{color}
\setmarginsrb{3 cm}{2.5 cm}{3 cm}{2.5 cm}{1 cm}{1.5 cm}{1 cm}{1.5 cm}

\title{Preguntas de reflexión}		% Título
\author{\centering Moreno Chávez Jesús Rodolfo}											% Autores
\date{08 de Febrero del 2017} %Aquí pueden cambiarla%					% Fecha de edición

\makeatletter
\let\thetitle\@title
\let\theauthor\@author
\let\thedate\@date										
\makeatother

%% Para que salga el título en todas las hojas
%\pagestyle{fancy}
%\fancyhf{}
%\lhead{\thetitle}
%\cfoot{\thepage}

\begin{document}

%%%%%%%%%%%%%%%%%%%%%%%%%%%%%%%%%%%%%%%%%%%%%%%%%%%%%%%%%%%%%%%%%%%%%%%%%%%%%%%%%%%%%%%%%

\begin{titlepage}
	\centering
    \vspace*{0.5 cm}
    \includegraphics[scale = 0.5]{unison}\\[0.5 cm]	% University Logo
    \textsc{\Large Universidad de Sonora}\\[1.0 cm]	% University Name
	\textsc{\Large División de Ciencias Exactas y Naturales}\\[0.5 cm]				% Course Code
	\textsc{\large Física computacional I}\\[0.5 cm]				% Course Name
	\rule{\linewidth}{0.2 mm} \\[0.4 cm]
	{ \huge \bfseries \thetitle}\\
	\rule{\linewidth}{0.2 mm} \\[0.5 cm]
	
	\begin{minipage}{\textwidth}
		\begin{flushleft} 
			\emph{\Large} \large \\
			\theauthor
			\end{flushleft}
	
		\begin{flushleft} 
			\emph{\Large Profesor:} \large \centering Carlos Lizárraga Celaya 	
			\end{flushleft}
	\end{minipage}\\[1 cm]
	{\large \thedate}\\[2 cm]
 
	\vfill
	
\end{titlepage}
%%%%%%%%%%%%%%%%%%%%%%%%%%%%%%%%%%%%%%%%%%%%%%%%%%%%%%%%%%%%%%%%%%%%%%%%%%%%%%%%%%%%%%%%%%%

\section*{¿Cuál es tu primera impresión del uso de bash/Emacs?}
No pensaba que fuera útil para la recopilación de datos.
\\

\section*{¿Ya lo habías utilizado?}
Sí, para escribir programas en lenguaje FORTRAN.
\\
\section*{¿Qué cosas se te dificultaron más en bash/Emacs? }
Saber cómo crear un script para descargar los datos.

\section*{¿Qué ventajas les ves a Emacs? }
El poder obtener los datos de todo el año sin la necesidad de copiar y pegar.
\\
\section*{ ¿Qué es lo que mas te llamó la atención en el desarrollo de esta actividad?}
La manerá rápida y eficiente de obtener los datos.
\\
\section*{ ¿Qué cambiarías en esta actividad?}
\section*{¿Que consideras que falta en esta actividad? }
\section*{¿Puedes compartir alguna referencia nueva que consideras útil y no se haya contemplado?}
\section*{¿Algún comentario adicional que desees compartir?}
      
       
         
        
       
       
        
       
         

 



\end{document}