\documentclass{article}
\usepackage[spanish]{babel}
\usepackage[utf8]{inputenc}
\usepackage[T1]{fontenc}
\usepackage{graphicx}
\graphicspath{ {images/} }
\usepackage[hidelinks]{hyperref}
\usepackage[usenames]{color}
\usepackage{enumerate} %para "enuerar" con puntitos
\title{\vspace*{2.5in} \Huge Estructura de la atmósfera}
\author{\LARGE Moreno Chávez Jesús Rodlofo \\ \\ \LARGE Profesor: Carlos Lizárraga Celaya}


\begin{document}
\maketitle
\newpage
En este documento se habla sobre cuál es la estructura  de la atmósfera de la Tierra, enfocado en sus características físicas.

\section*{Introducción}
La Tierra posee una mezcla de varios gases que la rodean, a estos gases que no escapan debido al campo gravitatorio de la Tierra (o en cualquier objeto celeste) se le conoce como atmósfera.\\
La atmósfera determina el tiempo y el clima, además de  proteger al planeta y proporcionar los gases que necesitamos los seres vivos.\\
En la Tierra, la actual mezcla de gases se ha desarrollado a lo largo de 4.500 millones de años. A lo largo de todo este tiempo, diversos procesos físicos, químicos y biológicos transformaron esta antigua atmósfera a hasta dejarla como la conocemos actualmente.\\  
En el presente resumen se dará a conocer cuál es la estructura y algunas de sus características más importantes de la atmósfera de la Tierra.


\newpage
La atmósfera es la capa de gas (principalmente nitrógeno y oxígeno) que
rodea la Tierra. En comparación con el diámetro aproximado de 12.000 kilóme-
trosm de la Tierra, la capa atmosférica es delgada, cerca del 99 por ciento de
todo el gas atmosférico está concentrado en los primeros 30 kilómetros desde la
superficie terrestre.\\
Esta fina capa de gas nos brinda el aire que respiramos, protege el planeta de la
amenaza destructora de los meteoros y escombros espaciales, nos protege de la
radiación ultravioleta proveniente del Sol y almacena la energía térmica gracias
a la cual la Tierra es habitable, tanto para el ser humano como para las demás
formas de vida. La atmósfera contiene además vapor de agua, un componente
de enorme importancia que contribuye a la formación de las nubes de agua y de
hielo que generan los distintos tipos de precipitación.\\
El vapor de agua también almacena y libera grandes cantidades de energía tér-
mica denominada calor latente que constituye la fuente de energía que estimula
el desarrollo de tormentas y huracanes La estructura de la atmósfera se compo-
ne en capas definidas por los cambios de temperatura que se producen con la
altitud.\\ \\
La atmósfera consiste en cuatro capas: la troposfera, la estratosfera, la mesosfera
y la termosfera.
\begin{figure}[h]
\centering
\includegraphics[width=7cm, height=9.63cm]{atmsf}
\caption{}
\end{figure}

\newpage
La troposfera contiene casi el 75 \% de la masa gaseosa de la atmósfera. El
nitrógeno constituye el 78 \% del volumen total de gas seco, y el oxígeno otro
21\%. El 1\% restante del volumen lo forman el argón, el neón, el helio, el hidró-
geno, el xenón y el dióxido de carbono. Este último, cuyas propiedades de efecto
invernadero son importantes, representa sólo una fracción del uno por ciento de
los gases que componen la atmósfera terrestre. \\
\subsection*{Troposfera}
\vspace*{0.1in}
La troposfera es la capa más baja de la atmósfera. La mayor parte de la masa (cerca de 75-80\%) de la atmósfera está en la troposfera. Casi todos los estados del tiempo (estado del tiempo es la condición en que se encuentra la atmósfera en un determinado momento y lugar) ocurren en esta capa. \\ La temperatura disminuye con la altura en la troposfera, una de las razones es que aunque la energía del sol desciende del cielo,es absorbido principalmente por el suelo. El suelo está liberando constantemente esta energía, como el calor en luz infrarroja, así que la troposfera se calienta realmente de la tierra para arriba, haciendo que sea más caliente cerca de la superficie y más frío más arriba. Esto mantiene ``mezclado`` el aire de la troposfera. \\

\begin{figure}[h]
\centering
\includegraphics[width=10cm, height=8.5cm]{tropo}
\caption{}
\end{figure}


\newpage

\subsection*{Estratosfera}
\vspace*{0.1in}
En esta capa la temperatura aumenta con la altura. Esto se debe a que en la estratosfera  se sitúa la capa de Ozono, es decir, la Ozonosfera. El ozono ($0_3$) es un gas estable que está caliente ya que absorbe los rayos ultravioleta (UV) del sol.

\subsection*{Mesosfera}
La mesosfera es la capa que está por encima de la estratosfera.
La temperatura disminuye con la altura al igual que como lo hace en la troposfera
Esta capa también contiene proporciones de nitrógeno y oxígeno similares a la troposfera,Excepto que las concentraciones son 1000 veces menos y hay poco vapor de agua allí, Así que el aire es demasiado delgado para que ocurra el tiempo.

\subsection*{Termosfera}
La termosfera es la capa superior de la atmósfera.En esta capa la temperatura aumenta con la altura porque está siendo calentada directamente por el sol.
También, a esta capa se denomina como ionosfera ya que los átomos y moléculas existentes se encuentran en forma de iones.
\\ \\ \\La atmósfera es parte importante de lo que hace posible que la Tierra sea habitable. Bloquea y evita que algunos de los peligrosos rayos del Sol lleguen a Tierra. Atrapa el calor, haciendo que la Tierra tenga una temperatura agradable. Y el oxígeno dentro de nuestra atmósfera es esencial para la vida
\newpage
\section*{Bibliografía}

\begin{itemize}
\item \url{http://climate.ncsu.edu/edu/k12/.AtmStructure}  a 28 de enero de 2017
\item \url{https://www.meted.ucar.edu/fire/s290/unit4_es/navmenu.php?tab=1&page=2.0.0}  a 28 de enero de 2017 
\item \url{http://recursos.cnice.mec.es/biosfera/alumno/3ESO/energia_externa/contenidos6.htm}  a 28 de enero de 2017

\item \url{http://www.ambientum.com/enciclopedia_medioambiental/atmosfera/ESTRUCTURA-DE-LA-ATMOSFERA.asp}  a 28 de enero de 2017
\item \url{http://www.windows2universe.org/earth/Atmosphere/layers.html&lang=sp}    a 28 de enero de 2017
\end{itemize}
\subsection*{Imágenes}
\begin{itemize}
\item Figura 1:\url{https://www.meted.ucar.edu/fire/s290/unit4_es/navmenu.php?tab=1&page=2.1.0}
\item Figura 2: \url{https://www.meted.ucar.edu/fire/s290/unit4_es/navmenu.php?tab=1&page=2.1.1}
\end{itemize}

\end{document}






